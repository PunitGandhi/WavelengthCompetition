\documentclass[../main/WavelengthCompetition.tex]{subfiles} 




\begin{document}





\subsection{The one-wavelength case}
The case that $q=1$ and $\delta=0$ is the simplest case to consider since there is only one characteristic wavelength to the problem, namely 1.  We set the nonlinearity parameter to $b=1.8$ as was done by Burke et. al for the original SHE.    We already see a very interesting bifurcation diagram here (Fig.~\ref{fig:bifurcationdiagram1}).
\FIGbifurcationdiagramA
The periodic branch bifurcating from the homogeneous solution at $r=0$ has 20 periods and will be called $P20$.  The first secondary bifurcation from $P20$, forms a snaking branch of one-pulse localized states like the ones shown in Fig.~\ref{fig:snaking1}.
\FIGsnakingA
This branch eventually reconnects to another periodic branch that does not have a constant amplitude and has 12 periods in the domain (See Fig.~\ref{fig:doubleperiod1}).
\FIGdoubleperiod
This periodic solution does not bifurcate from the homogeneous solution, but from the $P24$ periodic branch.  It also has a loop in it, where the solution structure inverts so that the small oscillation are on the top instead of the bottom. The bifurcation is  actually the third branch point along the $P24$ branch if followed starting from the homogeneous solution.  The previous two branches along $P24$ are quite messy with seemingly randomly positioned pulses (See Fig.~\ref{fig:snakingmess1} for an example).
\FIGsnakingmess  
Additionally, we have looked at the solutions formed at the next two bifurcation points along the $P20$ periodic branch.  The second branch snakes with two-pulse localized states while the third branch is messy with seemingly randomly distributed pulses.

Another example of where a  multipulse state branch connects to a secondary periodic state branch is shown in Fig.~\ref{fig:bifurcationdiagram2}.  In this case, a five pulse state grows into the secondary periodic state upon each pulse doubling in size.  Note that this situation is slightly different in that both branches come from the same primary periodic branch.  It is also interesting to note that another five pulse branch appears below the periodic ( or ten pulse) branch, but does not connect as far as I've followed it. The secondary periodic branch with 10 periods shown in this figure contains a loop near $r=0.3$ just as the secondary periodic branch with 12 periods did in Fig.~\ref{fig:bifurcationdiagram1}.  
\FIGbifurcationdiagramB

There are several periodic solutions branching from the homogenous state between $P20$ and $P24$.  By looking at the secondary branches from these periodic states, we hope to see a pattern start to emerge with the location of multipulse states.  Note that we only track multipulse states in which all pulses are identical.  The table below lists in order, the periodic solutions along the diagram from in order of increasing forcing.  We would like to note that we can also interpret the periodic solution shown in Fig.~\ref{fig:doubleperiod1} as a twelve pulse state that just happens to exactly fill the simulation domain.  There are several solutions of this type that can be seen as a multipulse state with half the number of pulses as the periodic branch from which it emerges, or as a periodic solution with twice the wavelength of the branch from which it emerges.  



\begin{table}[ht]
\caption{Cascade of secondary multipulse branches that bifurcate from periodic states.} % title of Table
\centering % used for centering table
\begin{tabular}{c | c c c c c c c c c c} % centered columns (4 columns)
\hline\hline %inserts double horizontal lines
 Branch \# &\multicolumn{10}{c}{ Primary branch}\\
\hline  %\\  %[0.5ex] % inserts table 
%heading
10 	&\textbf{10p} &??		&??		&??		&??		&??		&??		&??		&??		&??	\\
9	&      		&??		&??		&??	 	&??		&??		&??		&??		&??		&??	\\
8	&4p  		&		&		&??	 	&??		&??		&??		&??		&??		&??	\\
7	&      		&		&3p		&??	 	&\textbf{11p}	&??		&??		&??		&??		&??	\\
6	& 2p    		&		&		&\textbf{9p} 	&2p		&??		&??		&		&		&??	\\
5	&\textbf{5p}  	&		&\textbf{7p}	&2p	 	&		&		&		&4p		&		&??	\\
4	&\textbf{4p} 	&		&3p		&	 	&2p		&		&		&		&3p		&??	\\
3	&      		&		&		&\textbf{6p} 	&		&		&		&2p		&		&\textbf{12p}	\\
2	&\textbf{2p} 	&		&		&	 	&2p		&		&		&		&5p		&	\\
1	&\textbf{1p}  	&		&\textbf{3p}	&2p	 	&		&		&		&\textbf{8p}	&3p		&2p	\\
\hline % inserts single horizontal line
 	&$P20$	&$P19$	&$P21$	&$P18$	&$P22$	&$P17$	&$P23$	&$P16$	&$P15$	&$P24$ \\ [1ex] % [1ex] adds vertical spac
\hline %inserts single line
\end{tabular}
\label{table:multipulse} % is used to refer this table in the text
\end{table} 

Following the homogeneous solution branch, we see a series of periodic states bifurcate as the forcing increases.  Each of these branches is labeled by the number of wavelengths that fit into the domain, and the sequence is shown in Table \ref{table:multipulse}.  Secondary bifurcations occur along each of these periodic branches as they are followed away from the homogeneous solution.  Studying the solutions along these secondary branches shows a cascade of multipulse solutions that have a very regular pattern.  The first periodic branch, $P20$ has  one-pulse solution branching off of the first bifurcation point, and then a two pulse solution off of the second one.  The pattern continues as you continue up the branch, except that the nth bifurcation produces an n-pulse state only if n is a factor of the number of periods of  on the branch (namely 20).  The next periodic branch begins with a two-pulse state at the first bifurcation point and increases as one follows the periodic branch up.  The pattern continues so that the next branch starts with a three-pulse state and increases.  Again the multipulse states only exist as multipulse states if the number of pulses that should be there is a factor of the period.  There are other, more complicated multipulse states that appear when the number of pulses that should appear a a particular bifurcation point has greatest common factor with the number of periods on the homogeneous branch.  In this case, the number of pulses on the branch will be the greatest common factor and this can be understood as effectively simulating only a fraction of the domain.  For example, an 8 pulse state should appear at the 8th bifurcation along $P20$, but this does not happen since 8 is not a factor of 20.  Instead, a 4-pulse state appears, which could be considered an 8 pulse state on a 40 period domain.  So, in effect, we've only simulated half of the domain resulting in only half of the expected number of pulses.  These multipulse states are indicated on the table in not-bold font, and generally seem to be more complicated patterns than the standard multipulse states.  \todo[inline]{ Note that the question marks on the table indicate that these branches have not yet been computed.  This still needs to be done.  The other thing I should do is to mark on the table where the separation between the bifurcation points near the primary bifurcation and the bifurcation points near the saddle-node bifurcation. }  

This pattern of increasing number of pulses seems to hold for the secondary branch points on the lower part of the periodic branch, and then repeats itself starting near the saddle-node bifurcation of the branch. All of the secondary branches were followed from the $P20$ periodic branch up to $r=0.4$, and the multipulse states are summarized in table~\ref{table:P20}. \todo[inline]{It may be possible that the algorithm missed some branch points if they were too close together, so I should try to rerun the calculation with smaller stepsizes just to be sure.   I should also follow these branches further along to see where they connect to.}. The pattern for the lower branch exactly matches the pattern for the upper branch, where the first branch point corresponds to the twelfth. The corresponding branches do not, however, seem to connect (as far as I have followed them).  The fifth branch and 16th branch, for example, both have 5 pulse states, but the two branches do not seem to connect.  The fourth and 8th branches are 4 pulse states that connect.  Similarly, the 15th and 19th branches connect with each other, but do not seem to connect to the earlier 4 pulse branch. The one pulse state at the 12th branch point has is a pulse within a patterned state, and doesn't really grow by adding periods in the same way that the first one pulse solution does.  It might be a little bit of a stretch to call it a one pulse state.

\begin{table}[ht]
\caption{Secondary multipulse branches that bifurcate from the $P20$ periodic state.} % title of Table
\centering % used for centering table
\begin{tabular}{c | c } % centered columns (4 columns)
\hline\hline %inserts double horizontal lines
 Branch \# &  Multipulse state \\
\hline  %\\  %[0.5ex] % inserts table 
%heading

\hline % inserts single horizontal line
1 & 1p \\
2& 2p\\
4 & 4p\\
5 & 5p\\
6 & 2p\\
8 & 4p\\
10 &10p\\
\hline%inserts single line
12 & 1p\\
13 & 2p\\
15 & 4p\\
16 & 5p\\
17 & 2p\\
19 & 4p\\
21 & 10p\\
[1ex] % [1ex] adds vertical spac
\hline %inserts single line
\end{tabular}
\label{table:P20} % is used to refer this table in the text
\end{table} 

Another surprising result is seen in the first secondary branch from $P19$ - a snaking pattern  begins  after a messy start along the branch.  The snaking pattern involves 3 pulses of unequal size, and grows by first loosing fronts on the central pattern and then gaining fronts on all three patterns. \todo[inline]{I don't think front is the right word, but I hope it is clear what is meant.} The pattern eventually fills the domain, except that defects remains.  The branch becomes messy again at this point as the defects shift around and change slightly.  The snaking bifurcation diagram is shown in Fig.~\ref{fig:bifurcationdiagram19a}, and some sample solutions along the snaking region are shown in Fig.~\ref{fig:snaking19a}.
\FIGbifurcationdiagramC
\FIGsnakingB


%\subsection{$q_*=1.2$, $\delta=10^{-5}$}
\subsection{A first attempt at the two wavelength case}
We now move onto considering the case when there are two competing patterned states of different wavelengths.  We will still have a domain size such that the first natural wavelength has 20 periods, but will choose $q$ such that the second wavelength has 24 periods.  Ideally we would have liked for both wavelengths to go unstable at the same value of $r$, i.e. set $\delta=0$.  AUTO does not, for some reason, detect the $r=0$ bifurcation along the homogeneous branch in this case.  It might possibly that the function it uses is the product of the eigenvalues, and if two go unstable the product remains positive.  \todo[inline]{I still need to figure out how to analyze the case when $\delta=0$ with AUTO.  I am trying to use a feature that allows me to load a solution calculated from when $\delta=10^{-5}$, but then change the value to $\delta=0$.  I can't make it work properly yet but am working on it.}  Regardless, it is interesting to consider the case when one wavelength becomes unstable at a slightly smaller forcing than the other.  We start with the case when the wavelengths are close together and since we had seen a  between the $P20$ and $P24$ branches of the one wavelength case, we decided that $q_*=1.2$ would be a reasonable first attempt.   \todo[inline]{I used the value $q=1.20001$ which I got using NSolve in Mathematica.  I don't think it was calculated to machine precision, and I'm not sure if this will make a difference or not.  I  should calculate $q$ with more precision just in case by either figuring out how to do this with Mathematica or writing a solver to do it for me.  I think an iterative solver might be the best approach in the latter case.}

In this case, we two periodic branches bifurcating from near $r=0$ as expected.  The first one has 20 periods and bifurcates basically at r=0 while the second has 24 periods and bifurcates at around $r=10^{-6}$.  We started out with the nonlinearity parameter at $b=1.8$ as before, but found that the periodic branches did not go as subcritical as before.  We decided increase the nonlinearity in order to push the saddle-node bifurcation point on the periodic branches to lower values of $r$ so that there would be a comparable region of bistability in this case as there was in the $q=1$ case.  With this as motivation, we have set $b=2.0$ for the simulations in this subsection.  \todo[inline]{I think $b=2.1$ might actually be slightly closer to the other case, and I should redo these diagrams with this value}       

The first secondary bifurcation point along the $P20$ periodic branch produces  four pulse states that eventually grow to a periodic state with either 8 or 16 periods, the bifurcation diagram shown in Fig.~\ref{fig:bifurcationdiagramq12}. 
\FIGbifurcationdiagramD
The branch terminating on the left of the diagram corresponds to solutions with pulses that have a minimum at the center (shown in Fig.~\ref{fig:snakingq12p20aL}) and it terminates at a complicated periodic solution with 8 periods on the domain.
\FIGsnakingC
 The branch terminating on the right of the diagram, on the other hand, corresponds to solutions with a maximum at the center of each pulse (shown in Fig.~\ref{fig:snakingq12p20aR}) and terminates in a more simple periodic solution with 16 periods on the domain.
\FIGsnakingD
The first secondary bifurcation along the $P24$ branch produces a 3 pulse state that snakes until the pulses fill the entire domain.  The branch does not, however, seem to terminate at a periodic solution.  The Bifurcation diagram is shown in Fig~\ref{fig:bifurcationdiagramq12} and solutions along the branch are shown in Fig.~\ref{fig:snakingq12p24a}
\FIGsnakingE



\end{document}